\documentclass[10pt,twocolumn]{article}
\usepackage{oxycomps}
\usepackage [english]{babel}
\usepackage [autostyle, english = american]{csquotes}
\MakeOuterQuote{"}
\pdfinfo{
    /Title (Tutorial Report)
    /Author (Bryanna Hernandez)
}
\title{Designing a UX Survey}
\author{Bryanna Hernandez}
\affiliation{Occidental College}
\email{bhernandez2@oxy.edu}

\begin{document}

\maketitle

\section{Introduction}
As a Computer Science major with a concentration in worldbuilding, my aim for a comps project is to build a program that allows theatre set designers and other creatives such as writers to build three-dimensional scenes. The purpose of this tool is to become a useful visual aid for students to refine their artistic vision. In the initial stages of my project, I have imagined implementing tools such as a theatre stage selector, choosing between stage types like proscenium or theatre-in-the-round, adding in prop and actor libraries to which users can submit their own models, utilizing a timeline to plan out blocking and lighting, and including a budgeting tool that can be used to keep track of prop cost financially and temporally.

However, the most important aspect of my project remains creating an application that is relevant to the interests of those who would use it. While I have participated in multiple ballet and theatre productions and have watched plays in many different environments, I lack the experience of actually designing a set. Therefore, while the above features seem useful to me, I found a tutorial that would guide me through the process of creating a codesign survey with the intention of understanding how users would like to interact with a set design application, verifying which, if any, of these features they believe would useful, and other feature or ethical considerations I had yet to take into account. Additionally, since I have considered marketing this tool to screenwriters or authors, I need to find out if this product has any appeal for this demographic and if their needs overlapped or were significantly different enough to require me to focus on set designers above all else. 

A successful outcome of my tutorial would therefore allow me to answer all of the above questions by crafting a survey that identifies my target audience and gives me results that are comprehensive and inform me of the initial direction to start building my project. While a high number of users taking my survey would seem like an ideal quantitative result, the focus of this tutorial is collecting high-quality answers that are focused toward specific issues that need addressing, rather than a high number of vague answers from users outside my intended demographic. 
\section{Methods}
\subsection{Goal Refinement}
The first step in my tutorial was to properly outline my objectives for the survey. While I had an initial idea of what I wanted the survey to address, as explained in the introduction, I was asked to consider five questions that would allow me to further refine my goals. My main goal was to understand the user needs for an application that builds three-dimensional scenes. Therefore the users that I wanted to focus on were both frequent users of pre-existing set design applications, such as VectorWorks or SketchUp, as well as new users who were unable to find a satisfactory program or pay their subscriptions. 

The main types of questions that I was going to focus on were about two different categories. The first category that I wanted input on was regarding the features that I have already brainstormed, to identify if users were excited about these ideas or had suggestions about what to tweak. This section would be necessary to create a working prototype, as the application would need to contain at least basic design features that users would find useful. The second category was to identify user pain points from the other competitor applications so that I could make the effort to either avoid or solve these issues. The purpose of this section is to help differentiate my project and make sure that it would be a relevant contribution.   
\subsection{Tightening my Target Audience} 
Now that I had identified my user base, I needed to understand four aspects about the users that would help me craft questions accordingly: product awareness, interests, language, and region. For product awareness, I took this to mean their familiarity with other products that accomplish a similar task. Because I am targeting both frequent and new users, this means that I would likely want to separate my survey into two sections based on the users. Users who are familiar with other products will likely be able to answer more detailed questions, while new users are more likely to feel overwhelmed if questions are too in-depth.
 
By understanding target interests, I can create questions that are more engaging to the user base. For example, since I know I am appealing to users with interests in musical theatre, I can use this interest to design questions that place the user in the scenario of building a musical and target their specific behaviors while in a creative headspace. This will help to make the questions more interesting and allow me to have access to their thought process without necessarily having to conduct a one-on-one interview with each user. 

The language that I use in building my survey is important, as I want to create questions that imply a certain level of knowledge regarding set design, such as the types of stages or the equipment used in a production. Because this product is primarily marketed toward people in the field, using general language instead of industry jargon may be counterproductive in gaining specific and high-quality responses. 

While "Region" was originally targeting the geography of the user to understand how this may affect preferences, I slightly altered this question to consider the domain of the set designer. Whether the designer is taking into account middle school productions or attempting to put on a community play affects the level of quality and money that can be put into the design. However, if I am expanding my user base to include authors who are only using this tool as a visual aid, then there is no budget to consider. Preferences will differ depending on the purpose of the designer and their intended audience or use. 
\subsection{Crafting the Questions}
Now that I had refined my goals, it was finally time for me to write the questions to include in my survey. Because the tutorial is generalized to suit multiple needs, this section included many tips of what types on questions to avoid and how to phrase questions. The tutorial recommends starting with easy, uncontroversial questions to make the user feel more comfortable and gain a certain amount of trust. 

The tutorial recommends that for any questions on a ranked scale, all points should be properly labeled. Questions should also avoid addressing compound issues, and address each concern separately. It is also important to avoid question formats that are too complicated and intimidating, as it is more likely that users will not finish the survey. If a question is too time-consuming, for example ranking ten elements from best to worst, it is better to simplify the question, both for the respondent and the designer. In these cases, it is unlikely that it matters about the differences between 4th and 6th place, so it may be best to only ask for an unranked top three. From there, it is easy to analyze only the most popular elements for similarities and differences. For a similar reason, open-ended questions can be useful but typically better suited for in-person interviews. Therefore, it is better to stick with multiple-choice questions and add in short answer questions at the end of certain sections as a catch-all. In order to gain more honest responses, questions should not be leading, such as framing a certain function or feature as a good or bad idea. 

Taking into account all of these tips, as well as my findings from the previous sections, I started to create the questions. First, I decided to create a section that would separate users into frequent and experienced compared to more amateur but curious users. Within this section, I asked multiple choice questions such as "Do you have experience with set design programs such as Vectorworks or SketchUp?" with a follow-up question to select all similar tools the user has worked with, and a section for the user to provide any additional applications. I also needed some further questions to understand the user's experience with set design in general, so I added a multiple-choice range question to figure out how many productions the user has been involved in as a theatre tech. By using this term, I hoped this would naturally help to distinguish users more familiar with the industry. 

From this, I created two sets of questions suited to the user. For users who are inexperienced with set design tools, I avoided asking any questions about pain points. Instead, I wanted to understand more about their initial thought process when designing a set. I felt that it would be best to ask these questions in more detail, to encourage the user to think deeply about a scenario they are unfamiliar with. Because these types of questions are more open-ended, I split this section up into several short answer questions that would only require a few words in response to keep the user's interest without frustrating or overwhelming them. I made sure to do my best to explain any jargon, and I also picked out visual aids that I hoped would be informative for a less experienced user. In particular, I picked out images of the different types of stages. Instead of asking leading questions about the features I have brainstormed, I instead tried to ask questions about the user's opinions regarding the problem the feature is attempting to address. For example, since one of the features is a prop library, I asked about the user's preferences regarding using super-specific props or more generic geometric shapes, and to brainstorm the top 3 props they believe would be used in their production. At the end of this section, I asked an open-ended question about any comments or additional points in their process they would like to add.

The question set targeted towards more experienced users was much longer. I included a similar section as to the one above, guiding the user through a brainstorming process, though I spent less time explaining each concept. Additionally, this was the userbase to which I directed questions regarding pain points. For this section, I wanted to use the same questions for each of the programs that the user had selected, rather than curating a specific set of questions for each tool, because I believed this would allow me to compare the programs' strengths and weaknesses more generally, so I could combat issues that needed to be addressed more urgently. This set included questions like "Do you currently use this tool?" so as to understand if the issues in the program are dealbreakers and a ranking system of "How easy was this tool to learn?" to understand how intuitive the tool is. In general, my purpose is to create a program that is open to beginning set designers, so I am hoping to leverage questions such as these to create an all-purpose application that is a gateway for creatives interested in the medium. When asking about the strengths of the tool, I did some brief research beforehand to list certain tools the software offers to see if users frequently used those tools, and what made those tools desirable to them. I used this same technique to ask about their frustrations with the software as well.
\subsection{Choosing a Distribution Tool}
My final task before launching the survey was selecting the appropriate tool for my questions. The tutorial recommends Google Forms because it allows question branching, uses basic analytics, and is free to use. It also allows for images to be added to questions, and is generally a quick and intuitive tool that I have previous experience with. Because my user base is going to end up mostly being Occidental College students, this tool will also be useful in collecting user information such as email for any follow-up interviews I might want to conduct. Therefore, I decided to choose this tool.
\section{Evaluation}
To measure if I have successfully built this codesign survey, I would have to deploy the survey. I would want to measure it on the number of people that have completed it. If the number is less than five after a number of weeks and consistently reminding students, then I would likely believe this to be a failure because the survey would be too long and uninteresting to hold student interest for long enough to complete it. However, more than 10 respondents would be a success. Regarding the answers, data quality is the most significant measurement. Ideally, responses will have allowed me to refine my ideas regarding my proposed features and to see if there are any new features I feel are necessary to implement. While this would require more qualitative analysis, the best outcome for knowing if I have successfully built the survey would be to receive actionable results that allow me to prioritize certain aspects of my project. A failure in this regard would be responses that are too simple or generic. Answers should be specific and contain some detail, even if they are short.  
\section{Reflection}
The process of following this tutorial was rather open-ended. Instead of choosing to follow a tutorial that could teach me how to perform a certain task, I decided that gaining insight into the project would be the most useful, and it would allow me to consider if I felt that my comps project is useful and feasible. I have some concerns about how unique my program will be compared to more professional software. Because I have yet to deploy the survey, I still feel unable to properly judge if this idea will end up as my final comps project. However, due to my worldbuilding concentration, I find it difficult to imagine other projects that would be able to fulfill this requirement. Even if I decided to shift my focus, I still feel that following this tutorial will be useful since I would likely end up creating another application, and therefore being able to obtain user feedback is an important skill to have. 

\end{document}